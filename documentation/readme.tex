\documentclass[12pt]{report}
\usepackage{machine_programmer}
\begin{document}
%\dominitoc
%\faketableofcontents
\chapter{Readme}
%\minitoc
\section{Abstract}
Make specifications for your programs, and allow the computer to evolve the
solutions. 

\section{Ingredients}
This project has several components:

\begin{itemize}
\item binary/encoding for encoding Pyash into 32 byte tablets. (beta)
\item binary/clprobe for getting OpenCL info and compiling .cl files to check
for syntax errors (beta)
\item Machine programmer for evolving Pyash programs in OpenCL on GPU/CPU  (alpha)
\item OpenCL compatible virtual machine for Pyash, the SPEL core-language (alpha)
\item Compiler for converting Pyash byte-code to other languages like LLVM (concept)
\end{itemize}

\section{Progress}
As of Aug 2016, this is just a prototype under active development. 
It is expected that once the Machine Programmer can contribute to it's own code
base, that the rate of development will increase. 

\section{Tiny OpenCL Teaching}
Even now it has some useful AGPLv3, OpenCL code, which can be adapted to
other projects. 
check out the following files for a mini OpenCL overview:
\begin{lstlisting}[language=bash]
source/hello.c
source/hello.cl
source/generic.h
source/generic.c
\end{lstlisting}

can test with
\begin{lstlisting}[language=bash]
cd source
gcc generic.c hello.c -lOpenCL  -o hello # possibly also -L../library 
./hello
\end{lstlisting}

\end{document}

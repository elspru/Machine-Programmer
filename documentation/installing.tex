\documentclass[12pt]{report}
\usepackage{machine_programmer}
\begin{document}
%\dominitoc
%\faketableofcontents
\chapter{Installation}
%\minitoc
\section{Introduction}
This is a guide for how to install Machine Programmer on Ubuntu
Linux\cite{Ubuntu}.
Ubuntu Linux is perhaps the most popular Linux Distribution,
also it is free to download and install (though donations are welcome).

If you have a different operating system such as those made by Microsoft or
Apple, feel free to get it working in whatever method you use, and send
corresponding documentation so it can be added as an option.

The version of Ubuntu Linux this tutorial will be refering to is 16.04,
though at least for proprietary AMD drivers, you may have better luck on 14.04
or Fedora.

If you use a different version of Linux, I'm sure can adapt the package names to
those which are suitable for your distribution.  If you'd like to document how
you did it then can forward it and will add your manual also.

\section{Dependencies}
As with any package the most difficult part is getting the dependencies
installed.

needs autotools to compile
\begin{lstlisting}[language=bash]
apt-get install autoconf automake libtool build-essential
\end{lstlisting}

\subsection{OpenCL}
The hardest part of compiling this is probably installing OpenCL.
If you have that down, can simply:
\begin{lstlisting}[language=bash]
./autogen.sh && ./configure && make && binary/programmer
\end{lstlisting}

Otherwise if that doesn't work, then you probably need OpenCL,
so follow along below.

\subsection{CPU OpenCL Installationdownload opencl header packages}

If you'd like to take advantage of your CPU cores, or
you don't have an OpenCL compatible GPU, then you'll have to install 
Portable Open Compute Library or
\href{http://pocl.sourceforge.net/}{POCL}.

POCL has some dependencies on Ubuntu
\begin{lstlisting}[language=bash]
apt-get install libhwloc-dev zlib1g-dev libclang-dev libx11-dev ocl-icd-dev
cmake opencl-headers
\end{lstlisting}

Then if you are sure you don't have GPU drivers, as installing this package may
conflict with any GPU drivers you may install:
\begin{lstlisting}[language=bash]
apt-get install ocl-icd-opencl-dev
\end{lstlisting}

then compile POCL based on it's instructions, if it's being difficult, can try
the cmake part of the instructions, as cmake may give you additional
dependencies. 

Also note that you need to have likely the most recent OpenCL headers available, 
as of this writing, pocl-0.13 needs opencl 2.0 headers.

If you find that the dependencies I listed about are insufficient, then please
email me with how you got it working.
Otherwise once you have POCL can simply 
\begin{lstlisting}[language=bash]
./autogen.sh && ./configure && make && binary/programmer
\end{lstlisting}

\subsection{GPU  OpenCL Installation}

For all GPU's the supported version of OpenCL may differ from the latest
version. To avoid deprecation warnings and unexplained segmentation faults, 
can install the header files which are pertinent to your supported OpenCL version. 

You can find out what version of OpenCL your hardware supports either from it's
documentation, or by successfully installing OpenCL and then running the hello
scripts, which will also give you information about your available OpenCL
platforms. 

For your convenience I've included zip files of the headers in library/.
For example if your GPU supports OpenCL 1.1 then can install by doing:
\begin{lstlisting}[language=bash]
cd library/
unzip OpenCL-Headers-opencl11.zip 
mv OpenCL-Headers-opencl11 CL
sudo mv CL /usr/local/include/
\end{lstlisting}


\subsubsection{(ARM) Mali GPU's (O-Droid XU3/4)}

So far have tested it with the Mali OpenCL SDK, which works on the ODroid, an
open-hardware heterogenous processing SoC board.


To run it need to do
\begin{lstlisting}[language=bash]
apt-get install mali-x11 # and probably restart X server.
\end{lstlisting}


Because the standard opencl-headers provided by Ubuntu are 2.0+, and the
Mali-T628 only supports up to 1.1 you may have to copy the CL folder 
from Mali SDK to /usr/local/include, to avoid a bunch of deprecation
warnings and-or errors.
\begin{lstlisting}[language=bash]
sudo cp -rv include/CL /usr/local/include
\end{lstlisting}

Strangely enough, in order to get it to compile, and to get it to run requires
different versions of libOpenCL.so. However it seeems if POCL is installed, then
can simply compile now with:

\begin{lstlisting}[language=bash]
./autogen.sh && ./configure make && binary/programmer
\end{lstlisting}

Otherwise if POCL is Not installed the commands would be:

\textbf{Mali's libOpenCL.so installation}

In order to compile need the ones from
\href{https://developer.arm.com/products/software/mali-sdks/mali-opencl-sdk/downloads}
{Mali OpenCL SDK}

After downloading that package, extract it and fix up platform.mk to reflect
your current platform. For example:
\begin{lstlisting}[language=bash]
CC:=arm-linux-gnueabihf-g++
AR=arm-linux-gnueabihf-ar
\end{lstlisting}


then in the Mali SDK folder compile the libOpenCL.so
\begin{lstlisting}[language=bash]
cd lib/ && make
\end{lstlisting}
once you have a libOpenCL.so can put it into the machine-programmer's 
library/ folder.
\begin{lstlisting}[language=bash]
cp lib/libOpenCL.so  $MACHINE_PROGRAMMER_PATH/library/
\end{lstlisting}
\begin{lstlisting}[language=bash]
./autogen.sh && ./configure LDFLAGS=-L./library && make && binary/programmer
\end{lstlisting}

\subsubsection{Intel GPUs}
Intel includes onboard GPUs on a lot of motherboards, 
so chances are good that if you have an Intel CPU, that you may also have an
intel GPU, which you can take advantage of using the well functioning open
source beignet drivers.

\begin{lstlisting}[language=bash]
apt-get install beignet beignet-dev beignet-opencl-icd
\end{lstlisting}

Note this does not work with the AMD hardware that I've tested it with.

\subsubsection{Nvidia GPUs}
Unfortunately there are no libre drivers for Nvidia that have OpenCL support,
however there are proprietary drivers, which may work in certain cases. 
Note this means you can't use UEFI, common on modern laptops, you'll have to
disable it in the bios.  Also this could make breaking changes so I advise you
backup your data before attempting to install proprietary drivers. 

Before you begin, make sure your system is up to date;
\begin{lstlisting}[language=bash]
sudo apt-get update && sudo apt-get upgrade && sudo apt-get dist-upgrade;
sudo apt-get autoremove #cleans up extra packages
\end{lstlisting}

There are a variety of versions of the nvidia drivers, this is because they are
very finicky and they might not all work with your GPU card.  For instance I had
to try several, and spend several days testing, before I finally figured out
which ones worked. 

The best one for my GeForce GTX 960M, is unknown-361. I tried their recommended
nvidia-367 and nvidia-370 those didn't work, and nvidia-340 didn't even show any
picture on the screen (good thing I had ssh). 

So I would advise you backup your system, and install ssh, so you could ssh
tunnel into your computer and fix it in case the monitor stops working due to
incompatible drivers. 

Another good troubleshooting method when the screen is blank due to proprietary
drivers, is to hold the shift-key during boot-up as this can let you enter the
grub menu, where you can select "Advanced options" and use a recovery mode or
different kernel. 

Also may be a good idea to have a boot recovery disk or installation disk handy,
because installing proprietary drivers may make the system unbootable. 

%If you have a GeForce GTX 960M, then can do
%\begin{lstlisting}[language=bash]
%apt-get install nvidia-361 
%\end{lstlisting}

At first reboot after installing the drivers I noticed that the computer turned
itself off and on several times before loading properly, I'm guessing this is
normal for prorpeitary drivers. 

You may find it doesn't work well with the standard dev packs, something I was
struggling with for days.  But fortunately can install the open source ones 
from Intel.

\begin{lstlisting}[language=bash]
apt-get install beignet-opencl-icd
\end{lstlisting}


You may also run into it not using the proprietary drivers, 
for instance if you do 
\begin{lstlisting}[language=bash]
lsmod |grep nvidia
\end{lstlisting}
it may give you nothing. 

in this case may need to update-alternatives
\begin{lstlisting}[language=bash]
sudo update-alternatives --config x86_64-linux-gnu_gl_conf
\end{lstlisting}

This however wont work if you don't have any working proprietary drivers. 

I manged to get nvidia proprietary-drivers working for one or two boots, however
I was not able to reliably reproduce it.

\subsubsection{Other GPU's}
You would have to see what is available for your platform,
if you have success in getting OpenCL, then please write up a summary and email
me.

\section{Developing}

If you would like to help with development, need some additional packages.
\begin{lstlisting}[language=bash]
apt-get install clang-format git
\end{lstlisting}

\section{Contact}
Can email me at \href{mailto:streondj@gmail.com}{streondj@gmail.com}
for details.
\printbibliography
\end{document}

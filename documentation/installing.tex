\documentclass[12pt]{article}
\usepackage[margin=2cm]{geometry}
\usepackage{listings}
\usepackage{hyperref}
\usepackage{color}
\lstset{frame=tb,
  language=bash,
  aboveskip=3mm,
  belowskip=3mm,
  showstringspaces=false,
  columns=flexible,
  basicstyle={\small\ttfamily},
  numbers=none,
  numberstyle=\tiny\color{gray},
  keywordstyle=\color{blue},
  commentstyle=\color{green},
  stringstyle=\color{mauve},
  breaklines=true,
  breakatwhitespace=true,
  tabsize=2
}
\title{Machine Programmer Installation}
\author{Logan Streondj}
\begin{document}
\maketitle
\tableofcontents
\section{Introduction}
This is a guide for how to install it on Ubuntu Linux.
Ubuntu Linux is perhaps the most popular Linux Distribution,
also it is free to download and install (though donations are welcome).
So if you have the misfortune of having a Micrsoft or Apple computer, 
feel free to format it and install Ubuntu Linux,
or set up a dual boot, if you need to transition slowly.  

The version of Ubuntu Linux this tutorial will be refering to is 16.04 or
Xenial, the current long-term-support release. 

If you use a different version of Linux, I'm sure can adapt the package names to
those which are suitable for your distribution.  If you'd like to document how
you did it then can forward it and will add your manual also.

\section{Dependencies}
As with any package the most difficult part is getting the dependencies
installed.

needs autotools to compile
\begin{lstlisting}[language=bash]
apt-get install autoconf automake libtool build-essential
\end{lstlisting}

\subsection{OpenCL}
The hardest part of compiling this is probably installing OpenCL.
If you have that down, can simply:
\begin{lstlisting}[language=bash]
./autogen.sh && ./configure && make && binary/programmer
\end{lstlisting}

Otherwise if that doesn't work, then you probably need OpenCL,
so follow along below.

\subsection{CPU OpenCL Installation}

If you don't have an OpenCL compatible GPU, then you'll have to install 
Portable Open Compute Library or
\href{http://pocl.sourceforge.net/}{POCL}.

POCL has some dependencies on Ubuntu
\begin{lstlisting}[language=bash]
apt-get install libhwloc-dev zlib1g-dev libclang-dev libx11-dev ocl-icd-dev cmake
\end{lstlisting}

Then if you are sure you don't have GPU drivers, as installing this package may
conflict with any GPU drivers you may install:
\begin{lstlisting}[language=bash]
apt-get install ocl-icd-opencl-dev
\end{lstlisting}

then compile POCL based on it's instructions, if it's being difficult, can try
the cmake part of the instructions, as cmake may give you additional
dependencies. 

If you find that the dependencies I listed about are insufficient, then please
email me with how you got it working.

Otherwise once you have POCL can simply 
\begin{lstlisting}[language=bash]
./autogen.sh && ./configure && make && binary/programmer
\end{lstlisting}

\subsection{GPU  OpenCL Installation}

\subsubsection{(ARM) Mali GPU's (O-Droid XU3/4)}

So far have tested it with the Mali OpenCL SDK, which works on the ODroid, an
open-hardware heterogenous processing SoC board.

Strangely enough, in order to get it to compile, and to get it to run requires
different versions of libOpenCL.so

To run it need to do
\begin{lstlisting}[language=bash]
apt-get install mali-x11 # and probably restart X server.
\end{lstlisting}

In order to compile need the ones from
\href{https://developer.arm.com/products/software/mali-sdks/mali-opencl-sdk/downloads}
{Mali OpenCL SDK}

After downloading that package, extract it and fix up platform.mk to reflect
your current platform. For example:
\begin{lstlisting}[language=bash]
CC:=arm-linux-gnueabihf-g++
AR=arm-linux-gnueabihf-ar
\end{lstlisting}

then in the Mali SDK folder compile the libOpenCL.so
\begin{lstlisting}[language=bash]
cd lib/ && make
\end{lstlisting}
once you have a libOpenCL.so can put it into the machine-programmer's 
library/ folder.
\begin{lstlisting}[language=bash]
cp lib/libOpenCL.so  $MACHINE_PROGRAMMER_PATH/library/
\end{lstlisting}

Because the standard opencl-headers provided by Ubuntu are 2.0+, and the
Mali-T628 only supports up to 1.1 you'll also have to copy the CL folder 
from Mali SDK to /usr/local/include, to avoid a bunch of deprecation
warnings and-or errors.
\begin{lstlisting}[language=bash]
sudo cp -rv include/CL /usr/local/include
\end{lstlisting}

After that can go back into the project folder
\begin{lstlisting}[language=bash]
./autogen.sh && ./configure LDFLAGS=-L./library && make && binary/programmer
\end{lstlisting}

\subsubsection{Nvidia GPUs}
Unfortunately there are no libre drivers for Nvidia that have OpenCL support,
however there are proprietary drivers, which may work in certain cases. 

If you have 

\subsubsection{Other GPU's}
You would have to see what is available for your platform,
if you have success in getting OpenCL, then please write up a summary and email
me.

\section{Developing}

If you would like to help with development, need some additional packages.
\begin{lstlisting}[language=bash]
apt-get install clang-format git
\end{lstlisting}

\section{Contact}
Can email me at \href{mailto:streondj@gmail.com}{\nolinkurl{streondj@gmail.com}}
for details.
\end{document}
